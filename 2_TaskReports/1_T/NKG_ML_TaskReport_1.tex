\documentclass[11pt,usenames]{article}
\usepackage{srcltx,pdfsync}
%\usepackage[pdflatex=false,recompilepics]{gastex} 
\usepackage{amsmath,amssymb,amsfonts}
\usepackage{color}
\usepackage{hyperref, url}
\usepackage{geometry}
\usepackage{graphicx,subfigure,wrapfig}
\geometry{verbose,a4paper,tmargin=30mm,bmargin=50 mm,lmargin=25mm,rmargin=16mm}
\usepackage{ifthen}

\def\Lap{\ensuremath{\mathcal{L}}}
\usepackage{fancyhdr}
\pdfpagewidth 8.5in
\pdfpageheight 11in

\pagestyle{fancy}
\headheight 35pt

\usepackage{listings} % For inserting Code
\usepackage{color} %red, green, blue, yellow, cyan, magenta, black, white
\definecolor{mygreen}{RGB}{28,172,0} % color values Red, Green, Blue
\definecolor{mylilas}{RGB}{170,55,241}


\graphicspath{{./Images/}}

%%%%%%%%%%%%%
\usepackage[fancythm]{jphmacros2e}
\renewcommand{\footrulewidth}{0.5 pt}

\rhead{\small{ML Project Task 1 Report}}
\chead{}
\lhead{\small{CAP 6610}}
\lfoot{Ninad Gaikwad - 41482960}
\cfoot{\thepage}
\rfoot{}

\title{}

\date{}


\begin{document}
	
% Formatting for the required code	
	\lstset{language=Matlab,%
		%basicstyle=\color{red},
		breaklines=true,%
		morekeywords={matlab2tikz},
		keywordstyle=\color{blue},%
		morekeywords=[2]{1}, keywordstyle=[2]{\color{black}},
		identifierstyle=\color{black},%
		stringstyle=\color{mylilas},
		commentstyle=\color{mygreen},%
		showstringspaces=false,%without this there will be a symbol in the places where there is a space
		numbers=left,%
		numberstyle={\tiny \color{black}},% size of the numbers
		numbersep=9pt, % this defines how far the numbers are from the text
		emph=[1]{for,end,break},emphstyle=[1]\color{red}, %some words to emphasise
		%emph=[2]{word1,word2}, emphstyle=[2]{style},    
	}
% Formatting for the required code
	
	\begin{center}
		{\sc ML Project Task 1 Report}\\
		University of Florida \\
		Computer \& Information Science Engineering
		\vspace{0.5 cm}
	\end{center}
	
	{\large \begin{center}
			\textbf{Project - Task 1 Report}\\
			GAN and VAE for Handwriting Recognition
	\end{center}}
	
	
	%\newpage
	
	
	%\tableofcontents
	
	
	\newpage
	
	
	\section{Data Procurement:}\label{section:DataProcurement}
	\begin{itemize}
		\item \textbf{MINST:} \url{http://yann.lecun.com/exdb/mnist/}
		\item \textbf{Devnagri Character Dataset:} \url{http://www.iapr-tc11.org/mediawiki/index.php?title=Devanagari_Character_Dataset } \textit{[If time permits]}
	\end{itemize}	
	
	\section{Development Technologies:}\label{section:DevelopmentTechnologies}
	\begin{itemize}
	\item \textbf{Programming Language:} Python3.5
	\item \textbf{Platform:} The code will be developed as a Github project with automated documentation using pydoc.
	\item \textbf{Packages:} NumPy and Pandas will be used for data-handling. For machine learning Keras+TensorFlow and PyTorch will be used, so as to make comparison between the two packages in terms of performance, computational speed and ease of writing code [Intended to be a learning exercise of both the packages]. Dash and Plotly to create the final dash board to present results. 
	\end{itemize}		
	
	\section{Computational Resources:}\label{section:ComputationalResources}
	\begin{itemize}
	\item \textbf{Personal Laptop:} 64bit 16GB 8$\times$1.80GHz Windows 10 Machine
	\item \textbf{Lab Desktop:} 64bit 8GB 8$\times$1.80GHz Linux [Ubuntu 18.04.2 LTS] Machine
	\item \textbf{HiPerGator:} \textit{[If time permits]}
	\end{itemize}	
		
		
		%\bibliographystyle{IEEEtran}
		%\bibliography{{./BiBFolder/BibFile}}	
	
\end{document}

		\begin{table}[htpb]
	\caption{Performance comparison of baseline and proposed controller.}
	\label{tab:TableLabel}
	\begin{center}
		\begin{tabular}{|c|c|c|c|}
			\hline
			& Baseline & Proposed & a \\
			\hline
			Refrigerator temp. violation ($hours/Day$) & 7.1250 & 0.0416 & b\\
			\hline
			Secondary loads not served (\% $time$) & 57 & 48.63 & c\\
			\hline
			Secondary loads not served (\% $time$) & 57 & 48.63 & c\\
			\hline								
		\end{tabular}
	\end{center}
\end{table} 


\begin{table}[htpb]
	\caption{Test Table.}
	\label{tab:TableLabel1}
	\begin{center}
		\begin{tabular}{|c | c | c |}
			\hline
			\textbf{1} & \textbf{2} & \textbf{3} \\
			\hline
			\hline
			4 & 5 & 6 \\
			\hline
			\hline
			7 & 8 & 9 \\
			\hline							
		\end{tabular}
	\end{center}
\end{table}	


	This is the figure.

\begin{figure}[htpb]
	\centering
	\includegraphics[scale=0.25]{LatexIcon.png}
	\caption{Caption}
	\label{fig:FigureLabel1}
\end{figure}

The Fig~\ref{fig:FigureLabel1} is great.\\	


This is the figure.

\begin{figure}[htpb]
	\centering
	\includegraphics[scale=0.25]{LatexIcon.png}
	\caption{Caption}
	\label{fig:FigureLabel}
\end{figure}

The Fig~\ref{fig:FigureLabel} is great.












